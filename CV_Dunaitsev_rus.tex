\documentclass[letterpaper,11pt]{article}
\usepackage[T2A]{fontenc}
\usepackage[utf8]{inputenc}
\usepackage[russian, english]{babel}
\usepackage{latexsym}
\usepackage[empty]{fullpage}
\usepackage{titlesec}
\usepackage{marvosym}
\usepackage[usenames,dvipsnames]{color}
\usepackage{verbatim}
\usepackage{enumitem}
\usepackage[hidelinks]{hyperref}
\usepackage[english]{babel}
\usepackage{tabularx}
\usepackage{fontawesome5}
\usepackage{multicol}
\usepackage{graphicx}
\setlength{\multicolsep}{-3.0pt}
\setlength{\columnsep}{-1pt}
\input{glyphtounicode}

\RequirePackage{tikz}
\RequirePackage{xcolor}
\RequirePackage{fontawesome}
\usepackage{tikz}
\usetikzlibrary{svg.path}


\definecolor{cvblue}{HTML}{0E5484}
\definecolor{black}{HTML}{130810}
\definecolor{darkcolor}{HTML}{0F4539}
\definecolor{cvgreen}{HTML}{3BD80D}
\definecolor{taggreen}{HTML}{00E278}
\definecolor{SlateGrey}{HTML}{2E2E2E}
\definecolor{LightGrey}{HTML}{666666}
\colorlet{name}{black}
\colorlet{tagline}{darkcolor}
\colorlet{heading}{darkcolor}
\colorlet{headingrule}{cvblue}
\colorlet{accent}{darkcolor}
\colorlet{emphasis}{SlateGrey}
\colorlet{body}{LightGrey}

% Adjust margins
\addtolength{\oddsidemargin}{-0.6in}
\addtolength{\evensidemargin}{-0.5in}
\addtolength{\textwidth}{1.19in}
\addtolength{\topmargin}{-.7in}
\addtolength{\textheight}{1.4in}

\urlstyle{same}

\raggedbottom
\raggedright
\setlength{\tabcolsep}{0in}

% Sections formatting
\titleformat{\section}{
  \vspace{-4pt}\scshape\raggedright\large\bfseries
}{}{0em}{}[\color{black}\titlerule \vspace{-5pt}]

% Ensure that generate pdf is machine readable/ATS parsable
\pdfgentounicode=1

%-------------------------
% Custom commands
\newcommand{\resumeItem}[1]{
  \item\small{
    {#1 \vspace{-2pt}}
  }
}

\newcommand{\classesList}[4]{
    \item\small{
        {#1 #2 #3 #4 \vspace{-2pt}}
  }
}

\newcommand{\resumeSubheading}[4]{
  \vspace{-2pt}\item
    \begin{tabular*}{1.0\textwidth}[t]{l@{\extracolsep{\fill}}r}
      \textbf{\large#1} & \textbf{\small #2} \\
      \textit{\large#3} & \textit{\small #4} \\
      
    \end{tabular*}\vspace{-7pt}
}

\newcommand{\resumeSubheadingAdditional}[2]{
	\vspace{-2pt}\item
	\begin{tabular*}{1.0\textwidth}[t]{l@{\extracolsep{\fill}}r}
		\large#1 & \small #2 \\
		
	\end{tabular*}\vspace{-7pt}
}

\newcommand{\resumeSubSubheading}[2]{
    \item
    \begin{tabular*}{0.97\textwidth}{l@{\extracolsep{\fill}}r}
      \textit{\small#1} & \textit{\small #2} \\
    \end{tabular*}\vspace{-7pt}
}

\newcommand{\resumeProjectHeading}[2]{
    \item
    \begin{tabular*}{1.001\textwidth}{l@{\extracolsep{\fill}}r}
      \small#1 & \textbf{\small#2}\\
    \end{tabular*}\vspace{-5pt}
}

\newcommand{\lineSeparatedText}[2]{
	{\textbf{\large{\underline{#1}}}} $|$ {\large{\underline{#2}}}
}

\newcommand{\resumeExpItem}[2] {
	\item 
	\begin{tabular*}{1.001\textwidth}{l@{\extracolsep{\fill}}r}
	    #1 & {\small#2}
    \end{tabular*}\vspace{-5pt}
}

\newcommand{\resumeSubItem}[1]{\resumeItem{#1}\vspace{-4pt}}

\renewcommand\labelitemi{$\vcenter{\hbox{\tiny$\bullet$}}$}
\renewcommand\labelitemii{$\vcenter{\hbox{\tiny$\bullet$}}$}

\newcommand{\resumeSubHeadingListStart}{\begin{itemize}[leftmargin=0.0in, label={}]}
\newcommand{\resumeSubHeadingListEnd}{\end{itemize}}
\newcommand{\resumeItemListStart}{\begin{itemize}}
\newcommand{\resumeItemListEnd}{\end{itemize}\vspace{-5pt}}


\newcommand\sbullet[1][.5]{\mathbin{\vcenter{\hbox{\scalebox{#1}{$\bullet$}}}}}


\begin{document}

%----------HEADING----------


\begin{center}
    {\Huge \scshape Дунайцев Александр} \\ \vspace{1pt}
    Дата рождения: 16.10.1999 \\  City: Москва \\ \vspace{1pt}
    \small \href{tel:+79266490006}{ \raisebox{-0.1\height}\faPhone\ \underline{+7 (926)-649-00-06} ~} \href{dunaitsev.alexander@gmail.com}{\raisebox{-0.2\height}\faEnvelope\  \underline{dunaitsev.alexander@gmail.com}} ~ 
    %\href{https://linkedin.com/in/yourid}{\raisebox{-0.2\height}\faLinkedinSquare\ \underline{yourid}}  ~
    \href{https://github.com/dunaitseva}{\raisebox{-0.2\height}\faGithub\ \underline{dunaitseva}} ~
    \href{https://t.me/dunaitseva}{\raisebox{-0.2\height}\faTelegram\ \underline{dunaitseva}} ~
    %\href{https://www.hackerrank.com/yourid}{\raisebox{-0.2\height}\faHackerrank\ \underline{yourid}} ~
    %\href{https://codeforces.com/profile/yourid}{\raisebox{-0.2\height}\faPoll\ \underline{yourid}}
    \vspace{-8pt}
\end{center}


%-----------EDUCATION-----------
\section{ОБРАЗОВАНИЕ}
  \resumeSubHeadingListStart
  \resumeSubheading
  {МГТУ им. Н. Э. Баумана}{09.2019 -- 06.2023}
  {Бакалавр "Системы автоматизированного проектирования"}{Moscow, Russia}
  \resumeSubHeadingListEnd

\section{ДОПОЛНИТЕЛЬНОЕ ОБРАЗОВАНИЕ}
\resumeSubHeadingListStart
\resumeSubheadingAdditional
{Курс "Подготовительная программа C/C++" Технопарк МГТУ}{02.2020 -- 06.2020}
\resumeSubHeadingListEnd

\resumeSubHeadingListStart
\resumeSubheadingAdditional
{Курс "Системное адиминистирование Linux" Технопарк МГТУ}{02.2020 -- 06.2020}
\resumeSubHeadingListEnd

\resumeSubHeadingListStart
\resumeSubheadingAdditional
{Курс "Алгоритмы и структуры данных" VK Образование}{09.2021 -- 12.2021}
\resumeSubHeadingListEnd

\resumeSubHeadingListStart
\resumeSubheadingAdditional
{Курс "Системный архитекор" VK Образование}{02.2022 -- 06.2023}
\resumeSubHeadingListEnd
%\vspace{2pt}

\section{ПРОФЕССИОНАЛЬНЫЙ ОПЫТ}
\resumeSubHeadingListStart
\resumeExpItem {\lineSeparatedText{Sber Robotics Laboratory}{Стажер-разработчик ПО}}{05.2022 - ...}
\resumeItemListStart
\resumeItem{\normalsize{Разработка прошивок для семейства микроконтроллеров ESP32 с использованием фреймворка ESP-IDF.}}
\resumeItem{\normalsize{Исследование проблемы нестабильного Wi-Fi соединения в условиях сильной зашумленности окружения и поиск решения.
}}
\resumeItem{\normalsize{Реализация и тестирование WMN в качестве возможного решения нестабильного Wi-Fi соединения.}}
\resumeItem{\normalsize{Рефакторинг кодовой базы шаттловой системы. Разработка более гибкой программной архитектуры, позволившей упростить процесс сопровождения, обновления и доработки кода.}}
\resumeItemListEnd 
\resumeSubHeadingListEnd

%-----------PROJECTS-----------
\section{ПРОЕКТЫ}
    \vspace{-5pt}
    \resumeSubHeadingListStart
    \resumeProjectHeading
    {\href{https://github.com/dunaitseva/AES}{\textbf{\large{\underline{Реализация алгоритма AES}}} \href{Project Link}{\raisebox{-0.1\height}\faExternalLink }} $|$ \large{\underline{C++, GitHub Actions, GTest, CMake}}}{01.2021}
    \resumeItemListStart
    \resumeItem{\normalsize{Создал C++ библиотеку, которая предоставляет набор классов для использования алгоритма AES (в соответствии с FIPS 197 AES использует алгоритм симметричного шифрования Rijndael).}}
    
    \resumeItem{\textcolor{accent} {\href{https://github.com/dunaitseva/AES} {\underline{\normalsize{https://github.com/dunaitseva/AES}}}}}
    \resumeItemListEnd 
    \vspace{-13pt}
    
    \resumeProjectHeading
    {\href{https://github.com/dunaitseva/course_project_infosys_bmstu}{\textbf{\large{\underline{Веб-приложение "Hospital"}}} \href{Project Link}{\raisebox{-0.1\height}\faExternalLink }} $|$ \large{\underline{Python, Flask, HTML, CSS, Bootstrap, MySQL}}}{09.2021}
    \resumeItemListStart
    \resumeItem{\normalsize{В качестве практической задачи на курсовой работе, разработал веб приложение. Создал REST API для приложения, позволяющего заниматья управлением госпиталя. В ходе разработки приложения использовался архитектурный паттерн MVC.}}
    
    \resumeItem{\textcolor{accent} {\href{https://github.com/dunaitseva/course_project_infosys_bmstu} {\underline{\normalsize{https://github.com/dunaitseva/course\_project\_infosys\_bmstu}}}}}
    \resumeItemListEnd 
    
    \vspace{-13pt}
    
    \resumeProjectHeading
    {\href{https://github.com/dunaitseva/finite-diff-method}{\textbf{\large{\underline{Решатель уравнения теплопроводности пластины}}} \href{Project Link}{\raisebox{-0.1\height}\faExternalLink }} $|$ \large{\underline{C++, CMake, gnuplot, ANSYS}}}{05.2022}
    \resumeItemListStart
    \resumeItem{\normalsize{Создал библиотеку, которая реализует метод конечных разностей для решения нестационарной задачи уравнения теплопроводности для металической пластины произвольной формы и различными типами граничных условий.}}
    \resumeItem{\normalsize{Разработал набор классов для отрисовки анимации нагрева пластины.}}
    
    \resumeItem{\textcolor{accent} {\href{https://github.com/dunaitseva/finite-diff-method} {\underline{\normalsize{https://github.com/dunaitseva/finite-diff-method}}}}}
    \resumeItemListEnd 
    \resumeSubHeadingListEnd
    
    
\vspace{-12pt}

%-----------PROGRAMMING SKILLS-----------
\section{НАВЫКИ}
 \begin{itemize}[leftmargin=0.15in, label={}]
    \small{\item{
     \textbf{\normalsize{Языки программирования:}}{ \normalsize{C++, C, Python, Bash, SQL, \LaTeX}} \\
     \textbf{\normalsize{Технологии/Фреймворки:}}{\normalsize{ STL, Boost, GTest, CMake, Linux, Flask, Git, GitHub Actions, HTML, CSS, Bootstrap}} \\
    }}
 \end{itemize}
 \vspace{-15pt}
 
 %-----------ADDITION---------------
\section{ДОПОЛНИТЕЛЬНО}

$\sbullet[.75] \hspace{0.1cm}$ {{Опыт написания модульных и интеграционных тестов.}} \hspace{1.6cm} \\
$\sbullet[.75] \hspace{0.2cm}${{Опыт использования CI.}} \\
$\sbullet[.75] \hspace{0.2cm}${{Использование динамических (valgrind, sanitizers) и статических (cpplint, cppcheck, fbinfer, clang-tidy, etc.) инструментов анализа кода.} \hspace{2.6cm} \\
$\sbullet[.75] \hspace{0.2cm}${{Знание математики: линейная алгебра, методы оптимизации, вычислительная математика. Опыт в исследовании и реализации алгоритмов численных методов.}} \\
$\sbullet[.75] \hspace{0.2cm}${{Общие инженерные навыки. Опыт использования САПР (Siemens NX, Autodesk Inventor).}} \\
$\sbullet[.75] \hspace{0.2cm}${{Знание английского: Upper intermediate.}} \\
	
	

%\section{EMPLOYMENT PER WEEK}

%Ready to work 20-30 hours per week.

\end{document}